% internship.tex
\documentclass{beamer}
\usetheme[hideothersubsections,height=6mm]{Berkeley}
\setbeamertemplate{navigation symbols}{}

\usepackage{color}
\usepackage[ascii]{inputenc}
\usepackage{hyperref}


\title{Get An Internship}
\subtitle{Tips and Tricks}
\author{Chris Sims}
\date{25 Feb 2013}

\begin{document}

\frame{\titlepage}

\begin{frame}{Outline}
  \tableofcontents
\end{frame}


%-------------Why?------------%
\section{5 W's + H}
\begin{frame}\frametitle{What Are We Talking About?}

  \begin{itemize}
    \item Finding a suitable internship, either over the summer or part-time
          during the semester
    \item Applying for said internship
    \item Interviewing
    \item Having a fun and successful internship
  \end{itemize}

\end{frame}

\begin{frame}{Who Should Get An Internship?}
  An internship offers valuable experience and pay for anyone looking to get
  into a career of software development. \\
  \hfill

  You could probably get a decent job without an internship, but you can do
  better with one. Why not increase your chances to get your dream job? \\
  \hfill

  \begin{block}{But Should I Really?}
    Why not? It shows that you care about your career, you enjoy the work,
    you'll get paid to learn more, and it's usually fun!
  \end{block}
\end{frame}

\begin{frame}{Why Should I Go Through the Effort?}

  \begin{itemize}
    \item Experience
    \item Pay
    \item Required (499)
    \item Future job opportunities
    \item Networking
  \end{itemize}

\end{frame}

\begin{frame}\frametitle{When Should I Start Looking?}

Generally towards the start of Spring term for a summer internship. Employers
have started to realize that if they want the best intern candidate, they need
to start looking earlier in the year. Correspondingly, if you're looking for a
great internship, the best will continue to fill the longer you wait.\\

\vfill

Often summer internships will roll into the term, but if you're looking to start
mid-term with a new internship, the best time to start looking is now!

\end{frame}


\begin{frame}{Where to Look}
  There are a number of methods that companies use to list positions and to
  find candidates: \\
  \vfill

  \begin{columns}[c]
    \column{0.5\textwidth}
      Find positions
      \begin{itemize}
        \item Craigslist
        \item Monster.com
        \item Company websites
        \item Networking
      \end{itemize}

    \column{0.5\textwidth}
      Find candidates
      \begin{itemize}
        \item Internship/Career Fair
        \item LinkedIn
        \item Stack Overflow Careers
        \item Networking
      \end{itemize}

  \end{columns}
\end{frame}



\begin{frame}{How Do I Find a Good One?}
  There are a few items to take into consideration when looking:
  \begin{itemize}
    \item Is it a listing for an internship position?
    \item What languages do they mention in the intern posting?
    \item What does the company do? Does this interest you?
    \item Is there any information about what you'd be working on?
  \end{itemize}
\end{frame}


\section{Tools}
\begin{frame}\frametitle{Tools}

\begin{itemize}
  \item Craigslist - cheap, quick, easy, localized
  \item Monster.com
  \item Dice.com
\end{itemize}
\vfill
  \begin{block}{Want A LinkedIn Example?}
  \href{http://www.linkedin.com/pub/chris-sims/16/655/883}{My LinkedIn Profile}
  \end{block}
\end{frame}

\begin{frame}{LinkedIn}
  LinkedIn is a valuable tool that's used to find and identify
  candidates. \\
  \vfill
  Take the time to fill it out with:
  \begin{itemize}
    \item Experience
    \item Classes taken
    \item Projects completed
    \item Skills (languages, tools, etc.)
  \end{itemize}
  \vfill
  Remember to customize towards what you're looking for!

\end{frame}

\section{The R\'esum\'e}

\begin{frame}{The R\'esum\'e(s)}
  The r\'esum\'e is the main thing that conveys who you are to a potential
  employer. It's vital to keep a few things in mind when building it:

  \begin{itemize}
    \item Should have a cover letter if possible - not a generic one!
    \item Keep it updated - check it once a quarter, or just before a job
          search
    \item Keep it brief - should be 1 page
    \item Tailor it for different positions
    \item Don't list everything you've ever done - what's actually important?
  \end{itemize}

  Contact me if you want to see the r\'esum\'e that I use.
\end{frame}

\section{The Interview}
\begin{frame}{The Interview}
  \begin{columns}[t]
  \column{0.5\textwidth}
    Things to know beforehand
    \begin{itemize}
      \item What does the company do?
      \item Who am I interviewing with? What do they do?
      \item What position am I interviewing for?
    \end{itemize}
  \column{0.5\textwidth}
    Questions to ask/talk about during the interview
    \begin{itemize}
      \item Clarify the position that you're interviewing for
      \item What is the dev process like?
      \item What kind of tools/languages do they use?
      \item Any cool perks like seminars, training, etc?
    \end{itemize}
    \end{columns}

\end{frame}

\section{The Internship}
\begin{frame}{The Internship}
Obviously this depends greatly on where you work, but you'll notice a few things:
\begin{itemize}
  \item Some form of version control (git or svn most likely)
  \item Code can be messy, and documentation will likely be sparse or
        non-existent
  \item It will take time to really understand a project if it's been around for
        any period of time. Take the time to read any documentation, and get
        help with parts of the codebase you don't understand.

\end{itemize}

\end{frame}

\begin{frame}[Final Tips]\frametitle{Final Tips}
\begin{itemize}
  \item Ask questions!
  \item Documentation is your friend - find official documentation for the
        language(s) and the framework(s) and use them often
  \item Employers want to hire people that want to be there - you can show that
        in many different ways
  \item Take pride in your work
  \item Have fun!
\end{itemize}


\end{frame}

\end{document}
