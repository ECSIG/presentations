% emacs.tex
\PassOptionsToPackage{usenames,dvipsnames}{xcolor}
\documentclass{beamer}
\usetheme[hideothersubsections,height=6mm]{Berkeley}
\setbeamertemplate{navigation symbols}{}

\usepackage{listings}
\usepackage{hyperref}
\usepackage{tikz}
\usepackage{graphicx}

\usetikzlibrary{arrows,automata}

\definecolor{MyDarkGreen}{rgb}{0.0,0.4,0.0}

% For faster processing, load LaTeX syntax for listings
\lstloadlanguages{[LaTeX]TeX}%
\lstset{language=[LaTeX]TeX, frame=single, basicstyle=\small\ttfamily,
  keywordstyle=[1]\color{Blue}\bf, keywordstyle=[2]\color{Purple},
  keywordstyle=[3]\color{Blue}\underbar, identifierstyle=,
  commentstyle=\usefont{T1}{pcr}{m}{sl}\color{MyDarkGreen}\small,
  stringstyle=\color{Purple}, showstringspaces=false, tabsize=5,
  % Put standard LaTeX macros not included in the default
  % language here
  morekeywords={},
  % Put LaTeX macro parameters here
  morekeywords=[2]{},
  % Put user defined macros here
  morekeywords=[3]{}, morecomment=[l][\color{Blue}]{...},
  numbers=left, firstnumber=1, numberstyle=\tiny\color{Blue},
  stepnumber=5 }

\title{GNU Emacs}
\subtitle{The extensible, customizable editor}
\author{Chris Sims \\ chris@jcsi.ms}
\date{8 Oct 2013}

\begin{document}
\lstset{language=[LaTeX]TeX}
\frame{\titlepage}

\begin{frame}{Outline}
  \tableofcontents
\end{frame}


% -------------Why?------------%
\section{Why Not an IDE?}
\begin{frame}
  \frametitle{Why Emacs?}  GNU Emacs was originally written in 1984 by
  Richard Stallman, with origins dating back to 1976 in MIT's AI Lab.
  \begin{itemize}
  \item decades of work and refinement
  \item extensive help system and information online
  \item first-class plugin support
  \end{itemize}
\end{frame}

\begin{frame}
  \frametitle{Ok, So Why Not Vim or Sublime Text?}

  These are both great choices, but Emacs has:
  \begin{itemize}
  \item an integrated package manager
  \item proper subprocess support
  \item vast amount of packages available for almost any application
  \end{itemize}

\end{frame}

\section{Common Terminology}
\begin{frame}
  \frametitle{Basics}
  Terms:
  \begin{description}
  \item[\texttt{C-h t}] Hold down Control, hit {\tt h} then release, and then
    hit {\tt t}.
  \item[buffer] representation of a file or other resource
  \item[window] view on a particular buffer
  \item[mode] 1 major, {\tt n} minor modes define methods,
    redefine keybindings, and provide syntax highlighting (among
    other things)
  \end{description}
  Some things to remember:
  \begin{itemize}
  \item everything is a function ({\tt M-x})
  \item there are numerous help facilities to remind the user of functions,
    modes, variables, etc
  \end{itemize}
\end{frame}

\section{Getting Help}
\begin{frame}
  \frametitle{Immediate Help}
  \begin{description}
  \item[{\tt C-h t}] Emacs tutorial
  \item[{\tt C-h k}] help on a specific keybinding
  \item[{\tt C-h m}] docs on current modes
  \item[{\tt C-h f}] function docstring
  \item[{\tt C-h v}] help on a variable (including its current value)
  \item[{\tt C-h ?}] help (on the help system)
  \end{description}
\end{frame}
\begin{frame}
  \frametitle{Longer-Term Help}

  \begin{itemize}
  \item The Emacs Manual: {\tt C-h r}
  \item \url{http://emacswiki.org}: Chaotic, be cautious of
    outdated/wrong information
  \item {\tt .emacs}: Looking through other configurations can be
    quite fruitful
  \item Emacs reference cards: \url{http://www.gnu.org/software/emacs/refcards/index.html}
  \item Ask me!
  \end{itemize}
\end{frame}
\section{Extending Emacs}
\begin{frame}
  \frametitle{Plugins}
  {\tt package.el} was included in Emacs, starting with Emacs 24. It
  allows for installing packages (plugins) from a repository archive.
  \begin{description}
  \item[{\tt package-list-packages}] refresh repositories, list all
    available and installed packages
  \item[{\tt package-install}] install a specified package
  \end{description}
  \begin{block}{Adding Repositories}
    Marmalade and MELPA are two of the most diverse package
    repositories. MELPA builds from github source on a regular basis,
    and is your best source for new packages.
  \end{block}
\end{frame}

\section{Examples}
\begin{frame}
  \frametitle{\LaTeX}
  \begin{itemize}
  \item Auc\TeX\ is the definitive Emacs package for writing \LaTeX
  \item it's available in most package managers and from Emacs 24+
  \item Demo!
  \end{itemize}
\end{frame}

\begin{frame}
  \frametitle{Alternative Buffers}
  Buffers can be something other than just a view into a file
  \begin{itemize}
  \item terminal ({\tt eshell, shell})
  \item functional buffers ({\tt magit, dired})
  \item IRC ({\tt erc})
  \item REPL ({\tt R, nrepl})
  \end{itemize}
  Demo!
\end{frame}
\begin{frame}
  \frametitle{Org Mode}
  Org Mode is, at its simplest, an outlining tool.

  \begin{quote}
    ``Org-mode does outlining, note-taking, hyperlinks, spreadsheets,
    TODO lists, project planning, GTD, HTML and LaTeX authoring, all
    with plain text files in Emacs.'' --- Carsten Dominik, author Org
    Mode
  \end{quote}
  I use it for outlining, and keeping track of what I need to get done
  with the agenda.
  \vfill
  Demo!
\end{frame}
\begin{frame}
  \frametitle{My {\tt .emacs.d}}
  \begin{itemize}
  \item In newer versions of Emacs, the {\tt .emacs.d} directory (normally
  located at {\tt \textasciitilde/emacs.d}) contains the configuration (and package
  files) for Emacs.
  \item I use a fair number of packages, and have split some of the
  configuration out from {\tt init.el} into separate topical init
  files.
  \item Demo!
  \end{itemize}

\end{frame}

\begin{frame}
  \frametitle{Questions?}
  \begin{center}
  {\large Questions?} \\
  \vfill
  {\large Thanks!}
  \end{center}
\end{frame}
\end{document}

%%% Local Variables:
%%% mode: latex
%%% TeX-master: t
%%% End:
