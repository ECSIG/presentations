% hubot.tex
\documentclass{beamer}
\usetheme[hideothersubsections,height=6mm]{Berkeley}
\setbeamertemplate{navigation symbols}{}

\usepackage{fancyvrb}
\usepackage{color}
\usepackage[ascii]{inputenc}
\usepackage{hyperref}
\usepackage{examplep}


\title{Hubot}
\subtitle{Node.js IRC bot}
\author{Chris Sims}
\date{3 Dec, 2012}

\begin{document}

\frame{\titlepage}

\begin{frame}{Outline}
  \tableofcontents
\end{frame}


%-------------Coffeescript------------%
\section{Coffeescript primer}

\begin{frame}{Sample}
  Coffeescript is a language that compiles to Javascript.  It tends to be
  quite a bit more concise.

  \begin{columns}[c]
    \column{0.4\textwidth}
      { \tiny
      \input{coffee}
      }
    \column{0.6\textwidth}
      { \tiny
      \input{js}
      }
    \end{columns}

\end{frame}

%-------------Node.js-----------------%
\section{Node.js basics}

\begin{frame}{What is it?}
  \begin{itemize}
     \item Node is a software framework that allows you to write server-side applications
  in Javascript.

  \item Supposed to provide performant, asynchronous I/O in an easy-to-use package.

  \item Hubot was likely implemented on top of this because Coffeescript has been
  the \emph{hot thing} lately.

  \end{itemize}

  \end{frame}

  \begin{frame}{Hello World!}

  This simple example (pulled from \url{http://nodejs.org}) notifies the host OS
  that it wants to listen on a given address and port, and defines a callback for
  all requests.  In this case, it returns \texttt{Hello World}.

  \vspace{0.1\textheight}

  { \tiny
    \centering
  \input{node}
  }


\end{frame}

%------------Project structure---------%
\section{Project structure}

\begin{frame}{The project}
  \begin{columns}[c]
  \column{0.4\textwidth}
    \includegraphics[width=\textwidth]{tree.png}
  \column{0.6\textwidth}
    \begin{itemize}
      \item Most everything is pulled in as dependencies by NPM
      (Node Package Manager)
      \item Scripts can be pulled in by listing them in \texttt{hubot-scripts.json}
      \item \texttt{hubot} itself is pulled in as a dependency
    \end{itemize}
  \end{columns}
\end{frame}

%-----------Main class---------%
\section{Main class}

\begin{frame}{robot.coffee}
Methods of note:
\begin{itemize}
  \item \texttt{hear} - attempts to match words within a sentence
  \item \texttt{respond} - attempts to respond to directed requests
  \item \texttt{enter/leave} - performs action when a user enters/leaves the room
  \item \texttt{reply} - sends the built reply to the adaptor to handle
\end{itemize}


\end{frame}

%------------Sample plugin-------------%
\section{Practical}

\begin{frame}{Build a plugin}

Steps to add a simple plugin:
\begin{enumerate}
  \item Fork and clone \texttt{jcsims/hubot}
  \pause
  \item Create a topic branch - \texttt{git branch branchname}
  \pause
  \item Find a plugin that's close, copy and modify as needed
  \pause
  \item Commit changes, push to github
  \pause
  \item Initiate a pull request
  \pause
  \item Profit!!!
\end{enumerate}

\end{frame}


\end{document}
