% git_best_practices.tex
\documentclass{beamer}
\usetheme[hideothersubsections,height=6mm]{Berkeley}
\usefonttheme{structuresmallcapsserif}
\setbeamertemplate{navigation symbols}{}

\usepackage{color}
\usepackage[ascii]{inputenc}
\usepackage{hyperref}


\title{Git Best Practices}
\subtitle{Tips and Tricks}
\author{Chris Sims}
\institute{Engineering and Computer Science Interest Group \\
           URI Student ACM Chapter}
\date{6 March 2013}
\logo{\includegraphics[height=1cm]{logo_simple.png}}

\begin{document}

\frame{\titlepage}

\begin{frame}{Outline}
  \tableofcontents
\end{frame}


\section{The Commit Message}
\begin{frame}{The Unofficial Standard}
  Your commit message should communicate your work:
  \footnote{More details here:
  \href{http://tbaggery.com/2008/04/19/a-note-about-git-commit-messages.html}
  {http://tbaggery.com/2008/04/19/a-note-about-git-commit-messages.html}}

  \texttt{Capitalized, short (50 chars or less) summary}\\
  \vfill
  \texttt{More detailed explanatory text, if necessary.  Wrap it to about 72
    characters or so.  In some contexts, the first line is treated as the
    subject of an email and the rest of the text as the body.  The blank
    line separating the summary from the body is critical (unless you omit
    the body entirely); tools like rebase can get confused if you run the
    two together.
  }

\end{frame}


\section{Commit Early, Commit Often}
\begin{frame}\frametitle{Commit Early, Commit Often}
\begin{itemize}
  \item Commits and branches are cheap - use them!
  \pause
  \item Smaller commits allow for separating chunks of work, making later
        debugging or regression fixing easier
  \pause
  \item \texttt{git} has functionality that allows you to condense commits that
        you've made
  \pause
  \item A commit history with more discrete commits allows other contributors to
        better understand the work completed (Implement feature A vs. all the
        steps to get there)
\end{itemize}

\end{frame}


\section{Know Your History}

\begin{frame}{Backtracking}
  \texttt{git} has a wealth of tools available to recover past work. Some
  examples:
  \begin{itemize}
    \item \texttt{git log}: Review commits in the repository
    \pause
    \item \texttt{git reflog}: tracks all actions taken in the repository
    \pause
    \item Restore an older version of a file:
          \texttt{git checkout SHA -- path/to/filename}
    \pause
    \item Revert an entire commit (undo changes):
          \texttt{git revert SHA}
  \end{itemize}
\end{frame}

\begin{frame}{Changing History}
  Be careful rewriting history - generally the answer is never do this if this
  history is shared with others. Rewriting history that has been shared with
  others will often cause them to lose their work. If you haven't pushed your
  commits, there are some tools available:
  \begin{itemize}
    \item \texttt{git commit --amend}: Forgot to add something to that last
          commit, or typo in the commit message? This adds to the previous
          commit.
    \item \texttt{git pull --rebase}: Replays your commits on top of what you
          pull. This keeps a generally linear history, and avoids merge commits.
  \end{itemize}


\end{frame}

\section{Choose a Workflow}
\begin{frame}{Choose a Workflow}
  Pick a general workflow for things like adding features, fixing bugs, etc.
  Some general ideas:
  \pause
  \begin{itemize}
    \item Feature branches: take your work for a new feature into a branch, so
          that others can continue work in other areas. Pull changes from
          working branch to stay up-to-date.
    \pause
    \item Keep \texttt{master} at the latest stable release, with a \texttt{dev}
          branch for the latest.
    \pause
    \item Take a look at established workflows like \texttt{git-flow}
          \footnote{\href{http://nvie.com/posts/a-successful-git-branching-model/}
          {http://nvie.com/posts/a-successful-git-branching-model/}}
          or Pro Git branching models
          \footnote{\href{http://git-scm.com/book/ch3-4.html}
          {http://git-scm.com/book/ch3-4.html}}
  \end{itemize}

\end{frame}



\section{Final Tips}
\begin{frame}{Final Tips}
\begin{itemize}
  \item \texttt{git stash} should be part of your toolkit -
        \texttt{man git-stash}
  \pause
  \item Not quite sure about commands, or feeling nervous? Make an actual backup
        of your files and the \texttt{.git} directory, so you can restore if you
        mess it up.
  \pause
  \item Read the manual! There is documentation all over the place, and the Pro
        Git book
        \footnote{\href{http://git-scm.com/book/}{http://git-scm.com/book/}}
        is an outstanding resource.
  \pause
  \item Practice! Even if your repository never leaves your machine, you can
        still practice your git workflow and the various commands.
\end{itemize}


\end{frame}

\end{document}
