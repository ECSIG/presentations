% git_best_practices.tex
\documentclass{beamer}
\usetheme[hideothersubsections,height=6mm]{Berkeley}
\usefonttheme{structuresmallcapsserif}
\setbeamertemplate{navigation symbols}{}

\usepackage{color}
\usepackage[ascii]{inputenc}
\usepackage{hyperref}


\title{GitHub 101}
\subtitle{Social Git}
\author{Chris Sims}
\institute{Engineering and Computer Science Interest Group \\
           URI Student ACM Chapter}
\date{6 March 2013}
\logo{\includegraphics[height=1cm]{logo_simple.png}}

\begin{document}

\frame{\titlepage}

\begin{frame}{Outline}
  \tableofcontents
\end{frame}


\section{What is GitHub?}
\begin{frame}{What is GitHub?}
 GitHub is a web application designed around sharing code using \texttt{git}.
 It offers a few things that are pretty handy for software developers:

 \begin{itemize}
   \item Largest code host on the planet
   \item Free hosting for any open-source project (public repository)
   \item Makes contributing to other projects fairly painless
   \item Includes basic bug tracker and wiki for free
   \item Native apps for OSX, Windows, iOS, and Android
   \item Deals for students and student organizations
 \end{itemize}

\end{frame}


\section{GitHub Basics}
\begin{frame}{How Does It Work With Git?}
  \begin{itemize}
    \item Acts as a remote repository
    \item Gives anyone with an account pull access to public repositories
    \item Allows you to give fine-grained access to your repositories
    \item Offers a web interface to your repository with some additional
    functionality:
    \begin{itemize}
      \item Pull Requests - propose a change to an existing repository
      \item Social aspects - follow other repositories and developers
      \item Communicate about anything: a pull request, an issue, a line of
            code in a commit
    \end{itemize}
  \end{itemize}
\end{frame}

\section{Getting Started}
\begin{frame}{Create an account}
  You'll need an account for the practical portion, so if you don't already
  have a GitHub account, follow along:
  \begin{itemize}
    \item Head to \texttt{www.github.com}
    \item Click `Sign up for free'
    \item Choose a username, and use your URI email address if you want to get
          a free student account (gives 5 free private repositories)
    \item (Optional) Head to \texttt{www.github.com/edu} to request your free
          student account
    \item Remember your username and password! You'll need it for the practical
          portion
  \end{itemize}


\end{frame}

\begin{frame}{Practical}
  We're going to go through a short practical which will walk you through many
  of the commands and concepts that you've just learned. Head to: \\

  \vfill
  \texttt{http://try.github.com} \\

  \vfill
  and start the walkthrough. We'll be available to answer any questions you have.
  Look for the Advice box in the lower right for more information about
  \texttt{git}
\end{frame}


\end{document}
